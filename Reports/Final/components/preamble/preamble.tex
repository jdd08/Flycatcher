\title
{
	{\Huge \textsf{Flycatcher} \\[0.2cm]}
	{\large \textsf{Automatic unit test generation for JavaScript}\\[0.5cm]}
	\begin{figure}[h]
		\centering
		\includegraphics[scale=0.3]{./components/preamble/flycatcher.jpg}
	\end{figure}
}
\author
{	
	{\emph{Final Report}}\\[6.5cm]
	{\large \textbf{Jerome \textsc{de Lafargue}}}\\[0.2cm]
	\emph{supervised by}\\
	Susan Eisenbach \& Tristan Allwood\\[1cm]
	\textsc{\normalsize Department of Computing}\\
	\textsc{\large Imperial College London}
}


\date{}
\pagestyle{empty}
\maketitle

% insert empty page

\newpage
\mbox{}

\begin{center}
\textsc{\LARGE Abstract}\\[1.4cm]
\end{center}

The old adage \emph{``to err is human''} is more than manifest in the software world: programmers make mistakes. As a result, it is estimated that 50\% of software development time is spent on software testing alone. This has prompted many attempts in the past decades to fully \emph{automate} testing, through tools that generate tests autonomously. In addition to saving time, automatic test generation enables more systematic and unbiased testing, increasing the robustness and quality of programs.

JavaScript has recently risen in popularity and its status is changing from that of a browser scripting language to that of a more versatile and widely used programming language. The recency of this phenomenon means that JavaScript has not yet been adequately included in the automatic test generation effort. In response to this, we present \textsf{Flycatcher}, an automatic unit-test generation tool written for and in JavaScript. By proposing a tool that is capable of generating tests for a comprehensive subset of the JavaScript language, our aim is to fill the current gap for such a tool in the field of automatic test generation.

\newpage
\pagestyle{empty}
\mbox{}
\newpage
\mbox{}

\begin{center}
\textsc{\LARGE Acknowledgements}\\[1.4cm]
\end{center}

I wish to thank Susan Eisenbach and Tristan Allwood for taking the time off their busy schedules to discuss this project in our numerous meetings. Susan's extensive \textsf{DoC} experience was highly valuable and Tristan's technical feedback was always very helpful, at times enlightening.

\newpage
\mbox{}

% Format for the TOC and the LOF
\titleformat{\chapter}[display]
{\normalfont\Large\filcenter\sffamily\sc}
{\titlerule[1pt]%
\vspace{1pt}%
\titlerule
\vspace{1pc}%
\LARGE\MakeUppercase{\chaptertitlename} \thechapter}
{1pc}
{\titlerule
\vspace{1pc}%
\LARGE}

\addtocontents{toc}{\protect\thispagestyle{empty}}
\pagestyle{empty}
\tableofcontents
\cleardoublepage
\addtocontents{lof}{\protect\thispagestyle{empty}}
\pagestyle{empty}
\listoffigures
\cleardoublepage
\addtocontents{lof}{\protect\thispagestyle{empty}}
\pagestyle{empty}
\lstlistoflistings
\cleardoublepage

% Format for the chapter headings
\titleformat{\chapter}[display]
{\normalfont\Large\filcenter\sffamily\sc}
{%\titlerule[1pt]%
\vspace{1pt}%
\titlerule
\vspace{1pc}%
\LARGE\sc{\chaptertitlename} \thechapter}
{1pc}
{\titlerule
\normalfont
\vspace{1pc}%
\Huge}