\title
{
	{\Huge \textsf{Flycatcher} \\[0.2cm]}
	{\large \textsf{Automatic unit test generation for JavaScript}\\[0.5cm]}
	\begin{figure}[h]
		\centering
		\includegraphics[scale=0.3]{./components/preamble/flycatcher.jpg}
	\end{figure}
}
\author
{	
	{\emph{Final Report}}\\[6.5cm]
	{\large \textbf{Jerome \textsc{de Lafargue}}}\\[0.2cm]
	\emph{supervised by}\\
	Susan Eisenbach \& Tristan Allwood\\[1cm]
	\textsc{\normalsize Department of Computing}\\
	\textsc{\large Imperial College London}
}


\date{}
\pagestyle{empty}
\maketitle

% insert empty page

\newpage
\mbox{}

\begin{center}
\textsc{\LARGE Abstract}\\[1.4cm]
\end{center}

The old adage \emph{``to err is human''} is more than manifest in the software world: programmers make mistakes. As a result, it is estimated that 50\% of software development time is spent on software testing alone. This has prompted many attempts in the past decades to fully \emph{automate} testing, through tools that generate tests autonomously. In addition to saving time, automatic test generation enables more systematic and unbiased testing, increasing the robustness and quality of programs.

Dynamic languages have recently risen in popularity, as their flexibility suits today's fast-moving software industry. The growth of web applications has led to an increase in the use of JavaScript in particular. Formerly cast as a browser scripting language, JavaScript is becoming a more versatile programming language used for application development as well. This motivates research in automatic test generation for JavaScript, yet the existing work in that domain \cite{contract-driven,alshraideh2008complete,saxena2010symbolic,artzi2011framework} lacks autonomy and support for the class-based programming style which is becoming the norm in JavaScript development.

In this report, we present \textsf{Flycatcher}, an automatic unit-test generation tool written for and in JavaScript. We contribute to the field of automatic test generation by proposing a tool that is capable of successfully generating tests for a comprehensive subset of the JavaScript language. On top of providing the tester with a suite of unit tests, \textsf{Flycatcher} reports the errors found during the test generation process. Experimental evaluation shows that \textsf{Flycatcher} is capable of consistently achieving high code coverage with a selection of benchmark programs.

\newpage
\pagestyle{empty}
\mbox{}
\newpage
\mbox{}

\begin{center}
\textsc{\LARGE Acknowledgements}\\[1.4cm]
\end{center}

I wish to thank Susan Eisenbach and Tristan Allwood for taking the time out of their busy schedules to discuss this project in our numerous meetings. Susan's extensive \textsf{DoC} experience was highly valuable and Tristan's technical feedback was always very helpful, at times enlightening.

\newpage
\mbox{}

% Format for the TOC and the LOF
\titleformat{\chapter}[display]
{\normalfont\Large\filcenter\sffamily\sc}
{\titlerule[1pt]%
\vspace{1pt}%
\titlerule
\vspace{1pc}%
\LARGE\MakeUppercase{\chaptertitlename} \thechapter}
{1pc}
{\titlerule
\vspace{1pc}%
\LARGE}

\addtocontents{toc}{\protect\thispagestyle{empty}}
\pagestyle{empty}
\tableofcontents
\cleardoublepage
\addtocontents{lof}{\protect\thispagestyle{empty}}
\pagestyle{empty}
\listoffigures
\cleardoublepage
\addtocontents{lof}{\protect\thispagestyle{empty}}
\pagestyle{empty}
\lstlistoflistings
\cleardoublepage

% Format for the chapter headings
\titleformat{\chapter}[display]
{\normalfont\Large\filcenter\sffamily\sc}
{%\titlerule[1pt]%
\vspace{1pt}%
\titlerule
\vspace{1pc}%
\LARGE\sc{\chaptertitlename} \thechapter}
{1pc}
{\titlerule
\normalfont
\vspace{1pc}%
\Huge}